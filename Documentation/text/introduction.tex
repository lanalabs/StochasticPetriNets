\section*{Introduction}
\label{sec:intro}
Stochastic Petri nets can be used to model performance aspects of a Petri net.
\begin{itemize}
  \item What ratio of cases follows the upper path in my model?
  \item How long does it take to finish activity \emph{A}?
  \item How long will my current case take?
\end{itemize}
These and further analysis questions can not be answered by pure Petri nets. Therefore different ways to encorporate stochastic execution information into Petri nets were devised. See \cite{ciardo1994characterization} for an overview about different expressivity and analysis capabilities of stochastic Petri net formalisms that were proposed in literature.

In this plugin, we assume stochastic Petri nets to not be restricted to any special parametric form of duration distributions, i.e., we use simply \emph{SPN} in the convention of \cite{ciardo1994characterization}.

\subsubsection{Semantics}
For the execution semantics, we stick to the text-book semantics defined for GSPNs in~\cite{marsan1984class}. However, we allow generalized distributions, and do not restrict times to be exponentially distributed. We use the \emph{race semantics with enabling memory}\footnote{If you're interested, see the original discussion of different execution semantics in~\cite{marsan1989execution}.}, as initially defined in \cite{marsan1989execution}. This semantics allows to model time-outs in a process model, as well as parallelism. The same semantics is also used in the popular analysis tool TimeNET \cite{german1995timenet}. 

Transitions are separated into immediate transitions (firing without taking time) and timed transitions (delayed by a random duration) 
\begin{compactitem}
  \item a priority $p$ (used to specify sets of transitions that always fire before those with lower priority, $p$ is 0 for timed transitions, $p > 0$ for immediate transitions)
  \item a weight (used to probabilistically choose a transition from the group of highest priority transitions)
  \item a duration distribution (capturing the random delay of firing for timed transitions) 
\end{compactitem}

The semantics is as follows:
\begin{compactitem}
  \item determine the set of transitions $t_{enabled}$ that are candidates for being enabled (all transitions that have $> 1$ input tokens on their input places)
  \item determine the highest priority $p$ of these transitions.
  \item remove all transitions from $t_{enabled}$ with a lower priority than $p$
  \begin{compactitem}
    \item if $p > 0$, we choose an immediate transition randomly based on their weights.
    \item if $p = 0$,  the timed transitions race for the right to fire. Each pick a random duration from their distribution (sample a random value).
  \end{compactitem}
\end{compactitem}




 